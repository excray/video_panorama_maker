\section{Conclusion}

In this project, we have analyzed web videos as a panorama source. We implemented an automatic method to discover panorama sources from casual videos. 
The panorama sources should satisfy three properties, namely wide field-of-view, mosaicabilty and high image quality. Based on these properties, we formulate discovery of panoramas as an 
optimization problem that tries to achive an optimum balance between maximizing the visual quality of the resulting panoramas and maximizing the scenes that the panoramas cover. We have relaxed the original 
optimization problem to reduce the computational complexity. After finding these panorama sources, we stitch it to create scene panorama. 
In this project, we analyzed the feasibility of using web videos to create good quality panoramas. We have been successful in finding panoramas for a large number of casual videos.  We also found that the quality of panorama was dependent on the video quality
and the movement of the camera. Our method could be improved to include videos with skew, zoom and large camera rotations. Nevertheless, our method has been effecient to create interesting panoramas from web videos.