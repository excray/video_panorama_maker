\subsection{Image Stitching} \label{sec:ImageStitch}
Image stitching consisted of two main parts; aligning individual images together based off of the homographies 
and then blending the aligned images together into the final panorama. 

\subsubsection{Alignment}
Image alignment depends solely on the homography between two individual frames. In our implementation we 
calculate the homography between two consecutive frames and extract the associated translations; this allows
us to align the images in preparation for blending later on. From the homography calculation, we disregard
any knowledge of skew and rotation and thus maintain the rectangular proportion of each frame. This allows 
alignment to be simplified in addition to easier blending calculations explained in the next section.

\subsubsection{Blending}
Image blending is an important part of the stitching process. The method used in blending directly relates to
the quality of the final panorama produced. The blending process we utilized consisted of three steps, taking
a distance-weighted average for each pixel value, median-bilateral filtering, and incorporating the visual 
quality measurement. The overlapping area between frames consist of multiple pixels that need to blended 
together. The overall equation that is used is as follows:
\begin{equation}
I^P(X)=\displaystyle\sum_{k} w_k(x)\tilde{I}_k(x)/\displaystyle\sum_{k} w_k(x)
\end{equation}

Where $I_P(X)$ is the pixel value at index $x$ for panorama number $P$, $w_k(x)$ is the weighting for the 
pixel at index $x$ in frame number $k$, and $\tilde{I}_k(x)$ is the intensity of the pixel at index $x$ in 
it's corresponding frame $k$. The weighting for each pixel is the product of three calculations:
\begin{equation}
w_k(x) = w_{k1}(x)w_{k2}(x)w_{k3}(x)
\end{equation}

The final weighting for each pixel is the product of three calculations, $w_{k1}(x)$ is the shortest horizontal 
distance of the pixel $x$ from the nearest vertical edge; this is done to give more weight to those pixels to 
the center of the image than those at the edge. This calculation is represented by the following equation:
\begin{equation}
w_{k1}(x) =  \| \text{arg} \displaystyle\min_{y}\{\|y\|\; \tilde{I}_k (x+y) \text{ is invalid} \}\|
\end{equation}

Following which $w_{k2}(x)$ is the weighting as a result of a median-bilateral filter which takes into account
the average (median) value of all the pixels that overlap, from different frames, in the final panorama; this
is calculated using the equation shown below:
\begin{equation}
w_{k2}(x) = exp (-(\tilde{I}_k(x)-med(x))^2/\sigma^2)
\end{equation}

Lastly, the third term of the weight calculation, $w_{k3}(x)$, takes into account the individual quality of 
each frame that the pixel belongs to. The pixels that come frame a frame that is of a higher quality contribute
a higher weight to the final result; this is shown in the equation below:
\begin{equation}
w_{k3}(x) = exp (-(\gamma q_{bk}(k) + (1 - \gamma)q_{br}(k)))
\end{equation}

This blending method is applied to each color (RGB) in turn and the resulting image is the blended panorama.