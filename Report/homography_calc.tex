\subsection{Homography Calculation} \label{sec:HomoSec}

The homography is determined between consecutive frames in order to align images.  
SIFT is used to extract feature points of every frame.  
We integrated the SIFT demo program provided by David Lowe \cite{Lowe} in our technical solution to expedite the process.  
The RANSAC algorithm is used to obtain a robust homography from matching points between two sequential frames.  
$H$ shown in equation (\ref{eq:HMat}) is the constrained homography model we used in our RANSAC implementation.  
We assumed no skew or rotation in the homographies to simplify our image stitching implementation.  
As a result, there were only translations between frames.  The translations are used in discovering panoramas and image stitching.

\begin{equation} \label{eq:HMat}
H=
\left[ \begin{array}{ccc}
1 & 0 & t_{x} \\
0 & 1 & t_{y} \\
0 & 0 & 1 \end{array} \right]
\end{equation}

The homography error is used as a measure for the motion model accuracy.  
It is calculated as the average of second norm errors of each point.   
The error of each point is between the matching SIFT feature point and calculated point from the homography.  




